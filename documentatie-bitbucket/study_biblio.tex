\chapter{Bibliographic Research}\label{ch:studiubib}

\pagestyle{fancy}


{\color{blue}\noindent This chapter should take between 3 and 10 pages.\\}

Bibliographic research has as an objective the establishment of the references for the project, within the project domain/thematic. While writing this chapter (in general the whole document), the author will consider the knowledge accumulated from several dedicated disciplines in the second semester, 4$^{th}$ year (Project Elaboration Methodology, etc.), and other disciplines that are relevant to the project theme.

Each reference \textbf{must} be cited within the document. Please look at the examples below (depending on the project theme, the presentation of a method/application can vary).

Referințele are included in the Bibliography chapter.

References can by managed with \href{https://www.jabref.org/}{JabRef}, an application which can be downloaded from \url{https://www.jabref.org/#download}

Examples of what should be included in each type of reference can be found at \href{https://libguides.nps.edu/citation/ieee-bibtex}{here}.

About common errors found in online libraries of references you can read at \href{https://www.ece.ucdavis.edu/~jowens/biberrors.html}{here}


In Chapter 4 of ~\cite{Spizner2002}, Spitzner discusses the advantages and disadvantages of honeypot systems.

References will be included in the Bibliography section. The reference format must be IEEE, or similar. The introduction of new references in the Bibliography section, and their citation within the document text can be done manually (by obeying the format), but it is not recommended as it not easy to manage them, or by using the tools mentioned in the last paragraphs of this chapter.

In the Bibliography section, there are examples of references to conferences or workshops articles ~\cite{AntoniouSBB05}, journal ~\cite{AntoniouSBDB07}, and books ~\cite{russell1995artificial}, \cite{Spizner2002}. References to applications or online resources (web pages) must include at least a short relevant description in addition to the link~\cite{seleweb}, and other information is available (authors, year, etc.). References that contain only the link to the online resource will be placed in the page footer.

Each reference must be cited within the document text, see example below (depending
on the project theme, the presentation of a method/application can vary).

%În articolul~\cite{AntoniouSBDB07} autorii prezintă un sistem pentru ...
In paper~\cite{AntoniouSBDB07} the authors present a system for moving obstacle detection using stereo-vision and an estimation of own movement.

The method is based on ... autorii prezintă un sistem pentru detecția obstacolelor în mișcare folosind stereo viziune și estimarea mișcării proprii.
Metoda se bazează pe ...\textit{discuss the algorithms, data structures, functionality, specific aspects related to the project theme, etc….} Discussion: \textit{pros and cons}.

\section{A Section Name}
\section{Another Section Name}
\subsection{A Subsection Name}
{\color{red}DO NOT copy technology descriptions here}

